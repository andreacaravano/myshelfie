%
%                  Politecnico di Milano
%
%          Gruppo: AM34
%            A.A.: 2022/2023
%
% Ultima modifica: 02/05/2023
%
%     Descrizione: Prova Finale (Progetto) di Ingegneria del Software - A.A. 2022/23
%                  Peer review n. 2
%

\documentclass[a4paper,11pt]{article} % tipo di documento
\usepackage[T1]{fontenc} % codifica dei font
\usepackage[utf8]{inputenc} % lettere accentate da tastiera
\usepackage[english,italian]{babel} % lingua del documento
\usepackage{lipsum} % genera testo fittizio
\usepackage{url} % per scrivere gli indirizzi Internet e/o di riferimento nella pagina

\usepackage[hidelinks]{hyperref} % per modificare il comportamento dei collegamenti ipertestuali (+ leva colore attorno)

\usepackage[margin=0.7in]{geometry} % margine di pagina

\usepackage{graphicx} % per inserire immagini

\usepackage{fancyhdr} % per gestione intestazione e piè di pagina

\hypersetup{ % metadati di titolo e autore nel PDF
    pdftitle={Prova Finale di Ingegneria del Software - A.A. 2022/23},
    pdfauthor={Gruppo AM34}
}

\setlength{\parindent}{0pt} % rimuove l'indentazione del testo

\begin{document}
    \pagestyle{fancy}
    \fancyhead{}\fancyfoot{}
    \fancyhead[L]{\textbf{Prova Finale di Ingegneria del Software - A.A. 2022/23}}
    \fancyhead[R]{Gruppo AM34}
    \fancyfoot[C]{\thepage}

    \author{Andrea Caravano \and Biagio Cancelliere \and Alessandro Cavallo \and Allegra Chiavacci}
    \title{\textbf{\Large{Prova Finale di Ingegneria del Software - A.A. 2022/23\\Peer-review n. 2}}}
    \maketitle

    Valutazione della specifica del protocollo di comunicazione del gruppo \textbf{AM43}


    \section{Lati positivi}\label{sec:lati-positivi}
    \begin{itemize}
        \item Durante la fase di gestione del turno, risulta molto efficiente la costruzione
        di un messaggio preposto al solo aggiornamento del modello, che è infatti compatto.
        \item È prevista la gestione di una lobby di ammissione al gioco e un meccanismo comune di
        notifica per il raggiungimento degli obiettivi della partita.
    \end{itemize}


    \section{Lati negativi}\label{sec:lati-negativi}
    \begin{itemize}
        \item L'approccio seguito per la progettazione dei messaggi non è valido per
        entrambi i protocolli.

        In RMI, infatti, non vi è uno scambio di messaggi che sono invece sostituiti da chiamate
        di metodo, facendo uso di paradigmi che permettano di informare opportunamente i client
        dell'evoluzione del gioco (ad esempio come il binding degli oggetti del modello e/o metodi
        di callback).
        \item Non è chiara l'associazione tra un client e la propria partita. In tutti gli scambi,
        infatti, è previsto di informare il server sulla propria identità, anche in fase di disconnessione.

        Durante una partita, un client può fingersi un qualunque altro giocatore di qualunque
        altra partita e danneggiarne il flusso, essendo per giunta previsto un meccanismo di
        autorizzazione alla disconessione.
        \item Non sono specificate astrazioni per la gestione della ricezione dei messaggi
        della chat e di eventi eccezionali (quale appunto la disconnessione imprevista di un client).
        \item I diagrammi UML del modello e controller non trovano collegamento pratico con la specifica
        del protocollo di comunicazione.
        Quest'ultima, infatti, risulta spesso opaca nei confronti delle relazioni con il modello
        e il controller di gioco.
        \item Complessivamente, il documento ci è risultato poco curato e di difficile comprensione.
    \end{itemize}


    \section{Confronto tra le architetture}\label{sec:confronto-tra-le-architetture}
    \begin{itemize}
        \item Nella nostra architettura abbiamo previsto paradigmi di gestione degli scambi
        logicamente equivalenti ma distinti in maniera naturale sulla base del tipo di protocollo
        alla base della comunicazione.
        \item Abbiamo scelto di distinguere partite differenti attraverso identificativi della
        partita (in formato stringa).
        Un client può quindi scegliere la partita a cui accedere in fase di login, invece di essere
        ammesso alla prima disponibile (che è comunque una scelta attrettanto valida).
    \end{itemize}
\end{document}