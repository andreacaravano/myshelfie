%
%                  Politecnico di Milano
%
%          Gruppo: AM34
%            A.A.: 2022/2023
%
% Ultima modifica: 30/03/2023
%
%     Descrizione: Prova Finale (Progetto) di Ingegneria del Software - A.A. 2022/23
%                  Peer review n. 1
%

\documentclass[a4paper,11pt]{article} % tipo di documento
\usepackage[T1]{fontenc} % codifica dei font
\usepackage[utf8]{inputenc} % lettere accentate da tastiera
\usepackage[english,italian]{babel} % lingua del documento
\usepackage{lipsum} % genera testo fittizio
\usepackage{url} % per scrivere gli indirizzi Internet e/o di riferimento nella pagina

\usepackage[hidelinks]{hyperref} % per modificare il comportamento dei collegamenti ipertestuali (+ leva colore attorno)

\usepackage[margin=0.7in]{geometry} % margine di pagina

\usepackage{graphicx} % per inserire immagini

\usepackage{fancyhdr} % per gestione intestazione e piè di pagina

\hypersetup{ % metadati di titolo e autore nel PDF
    pdftitle={Prova Finale di Ingegneria del Software - A.A. 2022/23},
    pdfauthor={Gruppo AM34}
}

\setlength{\parindent}{0pt} % rimuove l'indentazione del testo

\begin{document}
    \pagestyle{fancy}
    \fancyhead{}\fancyfoot{}
    \fancyhead[L]{\textbf{Prova Finale di Ingegneria del Software - A.A. 2022/23}}
    \fancyhead[R]{Gruppo AM34}
    \fancyfoot[C]{\thepage}

    \author{Andrea Caravano \and Biagio Cancelliere \and Alessandro Cavallo \and Allegra Chiavacci}
    \title{\textbf{\Large{Prova Finale di Ingegneria del Software - A.A. 2022/23\\Peer-review n. 1}}}
    \maketitle


    \section{Lati positivi}\label{sec:lati-positivi}
    \begin{itemize}
        \item Ci sembrano ben pensate e ben suddivise tutte le possibili tipologie di controlli sulla gestione dei turni,
        oltre che di inizio e fine gioco.

        È anche ben curata l'allocazione di carte oggetto nel gioco e nella plancia, limitandole al numero massimo
        effettivamente previsto per ogni tipo.
        \item Ottima la suddivisione delle carte obiettivo comune, che differenziano gli algoritmi di controllo sulla base
        del pattern previsto dalla carta.
        \item Sono previsti nel controller tutti i metodi di notifica relativi all'andamento del gioco.
    \end{itemize}


    \section{Lati negativi}\label{sec:lati-negativi}
    \begin{itemize}
        \item Ci sembra limitante la scelta di utilizzare un ArrayList per rappresentare la plancia di gioco,
        in quanto richiede di sfruttare stratagemmi come delle liste di indici per rintracciare gli spazi adiacenti.

        Inoltre, risulterebbe difficoltoso trasporla graficamente all'utente.
        \item Risulta di primaria importanza la suddivisione tra il tipo di una carta oggetto e la sua rappresentazione:
        le carte sono distinte solo dal loro tipo e non è presente un attributo che ne indichi
        la rappresentazione nel gioco (le tre possibili immagini per tipo di carta).

        Oltretutto, la presenza di un tipo di carta \texttt{empty} aumenta la complessità.
        \item Potrebbe essere confusionaria l'eccessiva distribuzione dei parametri di gioco all'interno del
        modello, spesso richiamati attraverso molti metodi getter e setter (a volte sovrapposti tra entità differenti).
    \end{itemize}


    \section{Confronto tra le architetture}\label{sec:confronto-tra-le-architetture}
    \begin{itemize}
        \item Coerentemente con quanto visto a lezione, anche noi abbiamo scelto di raggruppare
        ulteriormente gli algoritmi di controllo delle carte obiettivo comune, mantenendo l'uso
        dello Strategy Pattern.
        \item Per la descrizione del protocollo di comunicazione, anche noi stiamo pensando alle principali
        categorie di messaggi scambiati tra client e server, per favorire l'interattività durante le fasi di gestione del gioco.
    \end{itemize}
\end{document}