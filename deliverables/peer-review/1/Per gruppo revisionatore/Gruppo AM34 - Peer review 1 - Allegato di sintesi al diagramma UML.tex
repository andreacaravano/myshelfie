%
%                  Politecnico di Milano
%
%          Gruppo: AM34
%            A.A.: 2022/2023
%
% Ultima modifica: 24/03/2023
%
%     Descrizione: Prova Finale (Progetto) di Ingegneria del Software - A.A. 2022/23
%                  Allegato di sintesi al diagramma UML di model e controller
%

\documentclass[a4paper,11pt]{article} % tipo di documento
\usepackage[T1]{fontenc} % codifica dei font
\usepackage[utf8]{inputenc} % lettere accentate da tastiera
\usepackage[english,italian]{babel} % lingua del documento
\usepackage{url} % per scrivere gli indirizzi Internet e/o di riferimento nella pagina

\usepackage[hidelinks]{hyperref} % per modificare il comportamento dei collegamenti ipertestuali (+ leva colore attorno)

\usepackage[margin=0.7in]{geometry} % margine di pagina

\usepackage{graphicx} % per inserire immagini

\usepackage{fancyhdr} % per gestione intestazione e piè di pagina

\hypersetup{ % metadati di titolo e autore nel PDF
    pdftitle={Prova Finale di Ingegneria del Software - A.A. 2022/23},
    pdfauthor={Andrea Caravano - Biagio Cancelliere - Alessandro Cavallo - Allegra Chiavacci}
}

\setlength{\parindent}{0pt} % rimuove l'indentazione del testo

\begin{document}
    \pagestyle{fancy}
    \fancyhead{}\fancyfoot{}
    \fancyhead[L]{\textbf{Prova Finale di Ingegneria del Software - A.A. 2022/23}}
    \fancyhead[R]{Andrea Caravano - Biagio Cancelliere - Alessandro Cavallo - Allegra Chiavacci}
    \fancyfoot[C]{\thepage}

    \author{Andrea Caravano \and Biagio Cancelliere \and Alessandro Cavallo \and Allegra Chiavacci}
    \title{\textbf{\Large{Prova Finale di Ingegneria del Software - A.A. 2022/23\\Peer-review n. 1\\Allegato di sintesi al diagramma UML}}}
    \maketitle

    \section*{}
    L'idea alla base della formulazione del modello (e di conseguenza, il controller) parte dal flusso di gioco.

    \medskip

    Si è scelto infatti di implementare parte della logica di gioco nelle classi rispettive, utilizzando un approccio a metà tra la totale chiusura rispetto a futuri aggiornamenti e la totale estendibilità delle stesse.

    \medskip

    Di ispirazione sono stati gli oggetti di tipo \texttt{Optional}, che abbiamo utilizzato profusamente all'interno del modello e della conseguente logica.

    \medskip

    Particolare attenzione è stata posta allo sviluppo di un tipo \texttt{BoardSpace} che, oltre a comprendere gli intuitivi concetti di coordinata,
    caratterizza le celle della plancia con un attributo indicante la propria utilizzabilità all'interno della struttura dati generale, nonché un parametro che
    tenga conto degli ‘‘spazi speciali’’, dipendenti da un numero specificato di giocatori.

    \medskip

    Il controller vuole inoltre rappresentare l'anello di congiunzione degli approcci seguiti in fase di implementazione del modello, offrendo
    metodi di gestione del protocollo di comunicazione, dei turni di gioco e della chat.

    \medskip

    Infine, è prevista la semplificazione della gestione di alcuni importanti eventi di interrupt del gioco, attraverso l'uso dei \texttt{Listener}.
\end{document}